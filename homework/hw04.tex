% Created 2024-04-18 Thu 14:45
% Intended LaTeX compiler: pdflatex
\documentclass{amsart}

\usepackage[utf8]{inputenc}
\usepackage[T1]{fontenc}
\usepackage{graphicx}
\usepackage{longtable}
\usepackage{wrapfig}
\usepackage{rotating}
\usepackage[normalem]{ulem}
\usepackage{amsmath}
\usepackage{amssymb}
\usepackage{capt-of}
\usepackage{hyperref}
\date{}
\title{Homework 4}
\hypersetup{
 pdfauthor={},
 pdftitle={Homework 4},
 pdfkeywords={},
 pdfsubject={},
 pdfcreator={Emacs 29.2 (Org mode 9.7-pre)}, 
 pdflang={English}}
\begin{document}

\maketitle
This homework is due by Friday, 3 May, 11:59pm on Gradescope.
\section{Problem 1  (16.3.2)}
\label{sec:org26015e5}

Determine the degrees of the splitting fields of the following polynomials over \(\mathbf{Q}\):
\begin{enumerate}
\item \(x^3-2\)
\item \(x^4-1\)
\item \(x^4+1\)
\end{enumerate}
\section{Problem 2 (16.6.2)}
\label{sec:orga5581e0}

Let \(K = \mathbf{Q}[\sqrt 2, \sqrt 3, \sqrt 5]\).
Determine \(\deg K / \mathbf{Q}\), prove that \(K/ \mathbf{Q}\) is a Galois extension, and determine its Galois group.
\section{Problem 3}
\label{sec:org1d20a37}

Let \(p\) be an odd prime number and \(K = \mathbf{Q}[\zeta_{p}]\).
Prove that \(K\) contains a unique degree 2 extension of \(\mathbf{Q}\).
\section{Problem 4}
\label{sec:org3e839b2}

Find quartic polynomials in \(\mathbf{Q}[x]\) whose Galois group is isomorphic to:
\begin{enumerate}
\item The Dihedral group \(D_4\) (of order 8)
\item The cyclic group \(C_4\)
\end{enumerate}

\textbf{Remark}: The general version of the above problem is a longstanding open problem called the \emph{Inverse Galois Problem}: given a finite group \(G\), does there always exist a polynomial in \(\mathbf{Q}[x]\) with Galois group isomorphic to \(G\)?
\section{Problem 5}
\label{sec:org5301092}

Let \(\delta \in \mathbf{Q}\) be such that \(\mathbf{Q}[\sqrt \delta]\) is the unique degree 2 extension of \(\mathbf{Q}\) contained in \(\mathbf{Q}[\zeta_{p}]\).
For \(p = 7\), find \(\delta\).
\section{Optional (Do not turn in)}
\label{sec:org2650781}

This is a continuation of the last problem.
You should now know the subfield \(\mathbf{Q}[\sqrt \delta] \subset \mathbf{Q}[\zeta_p]\) for \(p = 3, 5, 7\).
Based on this data, make a conjecture for an arbitrary odd prime \(p\).
(If you need more data, work out the case of \(p = 11\).)
Then try to prove the conjecture.
\end{document}
