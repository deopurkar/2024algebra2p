% Created 2024-05-03 Fri 13:21
% Intended LaTeX compiler: pdflatex
\documentclass{amsart}

\usepackage[utf8]{inputenc}
\usepackage[T1]{fontenc}
\usepackage{graphicx}
\usepackage{longtable}
\usepackage{wrapfig}
\usepackage{rotating}
\usepackage[normalem]{ulem}
\usepackage{amsmath}
\usepackage{amssymb}
\usepackage{capt-of}
\usepackage{hyperref}
\usepackage{fullpage}
\date{}
\title{Homework 5}
\hypersetup{
 pdfauthor={},
 pdftitle={Homework 5},
 pdfkeywords={},
 pdfsubject={},
 pdfcreator={Emacs 29.2 (Org mode 9.7-pre)}, 
 pdflang={English}}
\begin{document}

\maketitle
\emph{This homework is due by Friday, May 24, 11:59pm on Gradescope.  This is the last homework set, so I have given 3 weeks.}
\bigskip

The first three problems are about \emph{nested square roots}, namely complex numbers like \(\sqrt{\sqrt 2 + \sqrt{1 + \sqrt 3}}\).
More precisely, \(\alpha \in \mathbf{C}\) is a \emph{nested square root} if there exists a sequence of fields
\[ \mathbf{Q} = F_0 \subset F_1 \subset \cdots \subset F_n\]
such that each \(F_{i+1}/F_{i}\) is a quadratic extension and \(\alpha \in F_{n}\).
A nested square root is also called a \emph{constructible number} because these are precisely the complex numbers that can be constructed with a ruler and compass, starting with the two points \(0\) and \(1\).
\section{Problem 1  (16.9.3 modified)}
\label{sec:org6722f98}

Some nested square roots can be de-coupled to a linear combination of simple square roots.
For example, we have
\[ \sqrt {5 + 2\sqrt 6} = \sqrt 2 + \sqrt 3.\]
But some cannot be.
Prove that \(\alpha = \sqrt{1 + \sqrt 3}\) cannot be written as a sum
\[ \sqrt{a_{1}} + \cdots + \sqrt{a_{n}}, \quad a_i \in \mathbf{Q}.\]

\textbf{Hint}. Compare the Galois group of the minimal polynomial of \(\alpha\) over \(\mathbf{Q}\) and the Galois group of \(\mathbf{Q}[\sqrt{a_1}, \dots, \sqrt{a_n}] / \mathbf{Q}.\)
\section{Problem 2}
\label{sec:org385d7f9}

Let \(\alpha \in \mathbf{C}\) be a nested square root.
Let \(G\) be the Galois group of the minimal polynomial of \(\alpha\) over \(\mathbf{Q}\).
Prove that the order of \(G\) is a power of \(2\).

\textbf{Caution}. Make sure that the extension you are considering is Galois!
\section{Problem 3}
\label{sec:orgf6d994a}

Prove the converse to the problem before: if \(\alpha \in \mathbf{C}\) is such that its minimal polynomial over \(\mathbf{Q}\) has Galois group whose order is a power of 2, then \(\alpha\) is a nested square root.
As an application, show that if \(p\) is a prime number of the form \(2^n+1\), then \(\zeta_p\) is a nested square root.

With this, we have completed a proof of the following.\\
\textbf{Theorem}. For a prime number \(p\), the regular \(p\)-gon is constructible if and only if \(p\) has the form \(2^n+1\).

In this problem, you may use the following fact from group theory without proof. \\
\textbf{Theorem}. Let \(p\) be a prime and \(G\) a group of order \(p^{n}\) for \(n \geq 1\).  Then \(G\) contains a normal subgroup of index \(p\).
\section{Problem 4}
\label{sec:org9cdbcca}

Determine the Galois group of the polynomial \(x^6+3\) over the base fields
\begin{enumerate}
\item \(F = \mathbf{Q}\)
\item \(F = \mathbf{Q}[\zeta_3]\).
\end{enumerate}
\section{Problem 5 (16.12.7)}
\label{sec:org4b001ce}

Find a polynomial of degree \(7\) over \(\mathbf{Q}\) whose Galois group is \(S_7\).

\textbf{Hint}. Take inspiration from the construction in \emph{Artin} for degree 5.
\end{document}
