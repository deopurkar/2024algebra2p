% Created 2024-02-29 Thu 10:32
% Intended LaTeX compiler: pdflatex
\documentclass{amsart}

\usepackage[utf8]{inputenc}
\usepackage[T1]{fontenc}
\usepackage{graphicx}
\usepackage{longtable}
\usepackage{wrapfig}
\usepackage{rotating}
\usepackage[normalem]{ulem}
\usepackage{amsmath}
\usepackage{amssymb}
\usepackage{capt-of}
\usepackage{hyperref}
\date{}
\title{Homework 1}
\hypersetup{
 pdfauthor={},
 pdftitle={Homework 1},
 pdfkeywords={},
 pdfsubject={},
 pdfcreator={Emacs 29.1 (Org mode 9.7-pre)}, 
 pdflang={English}}
\begin{document}

\maketitle
This homework is due by Friday, 8 March, 11:59pm on Gradescope.

\section{Problem 1 (15.1.1)}
\label{sec:org3322e24}
Let \(R\) be an integral domain that contains a field \(F\) as a sub-ring.
Assume that \(R\) is finite dimensional when viewed as a vector space over \(F\).
Prove that \(R\) is a field.

\section{Problem 2 (15.2.1)}
\label{sec:org010d260}
Let \(\alpha\) be a complex root of the irreducible polynomial \(x^{3}-3x+4\) in \(\mathbf{Q}[x]\).
Find the inverse of \(\alpha^2+\alpha+1\) in the form \(a + b \alpha + c \alpha^{2}\) with \(a, b, c \in \mathbf{Q}\).

The particular polynomial and element are not important.
In fact, it is very likely that your method works in general.
But you do not have to explain a general method.


\section{Problem 3 (15.2.3)}
\label{sec:org83de333}
Let \(\beta = \omega \sqrt[3]{2}\), where \(\omega = e^{2\pi i / 3}\) and let \(K = \mathbf{Q}[\beta] \subset \mathbf{C}\).
Let \(k\) be a positive integer.
Prove that the equation
\[ x_1^2 + \cdots + x_k^2 + 1 = 0\]
has no solution with \(x_1, \dots, x_{k} \in K\).


\section{Problem 4 (15.3.5)}
\label{sec:orgb50f637}
For a positive integer \(n\), set \(\zeta_{n} = e^{2\pi i / n}\).
Find all values of \(n\) such that \(\zeta_{n}\) has degree at most 3 over \(\mathbf{Q}\).

You may use (without having to prove it) that for a prime number \(p\), the degree of \(\zeta_p\) over \(\mathbf{Q}\) is \(p-1\), and its minimal polynomial is
\[x^{p-1}+ x^{p-2} + \cdots + x + 1.\]

\section{Problem 5 (15.3.7)}
\label{sec:orgc49491f}
\begin{enumerate}
\item Is \(i\) in \(\mathbf{Q}[\sqrt[4]{-2}]\)?
\item Is \(\sqrt[3]5\) in \(\mathbf{Q}[\sqrt[3]2]\)?
\end{enumerate}

Justify your answers.
You may assume that \(x^3-5\) and \(x^3-2\) are irreducible over \(\mathbf{Q}\).  
If it helps you, feel free to assume that \(x^n \pm p\) is irreducible over \(\mathbf{Q}\) for any \(n\) and for any prime number \(p\).
\end{document}