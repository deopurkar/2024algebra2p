% Created 2024-03-08 Fri 16:12
% Intended LaTeX compiler: pdflatex
\documentclass{amsart}

\usepackage[utf8]{inputenc}
\usepackage[T1]{fontenc}
\usepackage{graphicx}
\usepackage{longtable}
\usepackage{wrapfig}
\usepackage{rotating}
\usepackage[normalem]{ulem}
\usepackage{amsmath}
\usepackage{amssymb}
\usepackage{capt-of}
\usepackage{hyperref}
\date{}
\title{Homework 2}
\hypersetup{
 pdfauthor={},
 pdftitle={Homework 2},
 pdfkeywords={},
 pdfsubject={},
 pdfcreator={Emacs 29.2 (Org mode 9.7-pre)}, 
 pdflang={English}}
\begin{document}

\maketitle
This homework is due by Friday, 22 March, 11:59pm on Gradescope.
\section{Problem 1 (15.4.1)}
\label{sec:org6ab0fa2}
Let \(K = \mathbf{Q}(\alpha)\) where \(\alpha\) is a complex root of \(x^3-x-1\).
Determine the irreducible polynomial for \(\gamma = 1 + \alpha^2\) over \(\mathbf{Q}\).
\section{Problem 2 (15.5.2(a))}
\label{sec:org0e2271c}
For this problem, first go through Section 5 (Construction with Ruler and Compass) to understand the proof of the following theorem (converse of what we did in class).

\bigskip

\noindent
\textbf{\textbf{Theorem:}} Suppose the coordinates of a point \(p\) lie in a field \(F = F_n\) such that there exists a chain of fields
\[ \mathbf{Q} = F_0 \subset F_1 \subset \dots \subset F_n\]
with \(\deg (F_{i+1} / F_i) = 2\) for all \(i\).
Then \(p\) is constructible by ruler and compass starting with \((0,0)\) and \((0,1)\).

\bigskip

Prove that a regular 5-gon is constructible by ruler and compass.
That is, prove that \((\cos (2\pi/5), \sin (2\pi/5))\) is constructible by ruler and compass starting with \((0,0)\) and \((0,1)\).
\section{Problem 3 (15.6.2 modified)}
\label{sec:orgd8acca5}
For this problem, first understand the proof of Proposition 15.3.3.

\bigskip

\noindent
\textbf{\textbf{Proposition:}} Let \(F\) be a field of characteristic not equal to 2.  Then every quadratic extension \(K/F\) can be written as \(K = F(\delta)\) where \(\delta^2 \in F\).

\bigskip

Let \(m, n \in \mathbf{Z}\).
Determine when \(\mathbf{Q}(\sqrt m)\) and \(\mathbf{Q}(\sqrt n)\) are isomorphic.
\section{Problem 4 (15.10.1)}
\label{sec:orgd58f874}
Prove that the subset of \(\mathbf{C}\) consisting of the algebraic numbers is algebraically closed.
\section{Problem 5 (15.7.8)}
\label{sec:orgdeb7abb}
The polynomials \(f(x) = x^3 + x + 1\) and \(g(x) = x^3 + x^2 + 1\) are irreducible over \(\mathbf{F}_2\).
Let \(K = \mathbf{F}_2[x]/(f(x))\) and \(L = \mathbf{F}_2[y]/g(y)\).
Describe explicitly an isomorphism from \(K \to L\).
Determine the number of isomorphisms from \(K \to L\).
\end{document}
