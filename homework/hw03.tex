% Created 2024-03-26 Tue 15:09
% Intended LaTeX compiler: pdflatex
\documentclass{amsart}

\usepackage[utf8]{inputenc}
\usepackage[T1]{fontenc}
\usepackage{graphicx}
\usepackage{longtable}
\usepackage{wrapfig}
\usepackage{rotating}
\usepackage[normalem]{ulem}
\usepackage{amsmath}
\usepackage{amssymb}
\usepackage{capt-of}
\usepackage{hyperref}
\date{}
\title{Homework 3}
\hypersetup{
 pdfauthor={},
 pdftitle={Homework 3},
 pdfkeywords={},
 pdfsubject={},
 pdfcreator={Emacs 29.2 (Org mode 9.7-pre)}, 
 pdflang={English}}
\begin{document}

\maketitle
This homework is due by Friday, 5 April, 11:59pm on Gradescope.
\footnote{The deadline is during teaching break, but I will have office hours as usual.}
\section{Problem 1 (15.7.6)}
\label{sec:org0a29665}

Factor the polynomial \(x^{16}-x\) over a field of size \(4\) and a field of size \(8\).
\section{Problem 2 (Presentations)}
\label{sec:org5f9268e}

Let \(R \subset S\) be an inclusion of rings.
Suppose we have an isomorphism
\[ S \cong R[x_1, \dots, x_n]/I, \]
where \(x_1, \dots, x_n\) are variables and \(I \subset R[x_1, \dots, x_n]\) is an ideal.
Such an isomorphism is called a \emph{presentation} of \(S\) over \(R\).

Let \(A\) be another ring and suppose a ring homomorphism \(i \colon R \to A\) is given.
A presentation of \(S\) over \(R\) gives us all the ways of extending \(i\) to a ring homomorphism \(S \to A\).
This is because a ring homomorphism \(R[x_1,\dots,x_n] \to A\) extending \(i\) is determined uniquely by the images of \(x_1, \dots, x_n\) and such a homomorphism is well-defined modulo \(I\) if and only if it sends \(I\) to \(0\).

\begin{enumerate}
\item Find a presentation for \(\mathbf{Q}[\sqrt[3]{2}]\) over \(\mathbf{Q}\).
Use it to determine all homomorphisms
\[ \mathbf{Q}[\sqrt[3] 2] \to \mathbf{C}.\]
What are the images of these homomorphisms?

\item Do the same for \(\mathbf{Q}[\sqrt 2, \sqrt 3]\) over \(\mathbf{Q}\).
\end{enumerate}
\section{Problem 3 (Automorphisms 1)}
\label{sec:org61a346f}

Let \(p\) be a prime number and set \(\zeta_p = e^{2\pi i / p}\).
Let \(F = \mathbf{Q}[\zeta_p]\).
Find all automorphisms
\[ \phi \colon F \to F.\]
Describe the automorphism group \(\operatorname{Aut}(F)\).
(This is the group consisting of automorphisms \(F \to F\) with composition as the group law.)
\section{Problem 4 (Automorphisms 2)}
\label{sec:orgf3e5b55}

Let \(F = \mathbf{Q}[\zeta_p]\), as before.
Let \(K = F[2^{1/p}]\).
Find all automorphisms \(\phi \colon K \to K\).
How many are there?
How many restrict to the identity on \(F\)?
\section{Problem 5}
\label{sec:org0c14ce3}

Let \(K\) be a field of size \(p^n\).
How many elements of \(K\) are perfect squares?
Generalise your answer to perfect \(d\)-th powers.
\end{document}
