% Created 2024-03-19 Tue 10:50
% Intended LaTeX compiler: pdflatex
\documentclass[12pt]{amsart}

\usepackage[utf8]{inputenc}
\usepackage[T1]{fontenc}
\usepackage{graphicx}
\usepackage{longtable}
\usepackage{wrapfig}
\usepackage{rotating}
\usepackage[normalem]{ulem}
\usepackage{amsmath}
\usepackage{amssymb}
\usepackage{capt-of}
\usepackage{hyperref}
\author{2024 Algebra 2}
\date{}
\title{Workshop 1}
\hypersetup{
 pdfauthor={2024 Algebra 2},
 pdftitle={Workshop 1},
 pdfkeywords={},
 pdfsubject={},
 pdfcreator={Emacs 29.2 (Org mode 9.7-pre)}, 
 pdflang={English}}
\begin{document}

\maketitle
\section{Degree of \(\mathbf{Q}(\cos (2\pi/p))\)?}
\label{sec:org8b72f22}

\noindent
Let \(p\) be a prime number.
What is the degree of \(\mathbf{Q}(\cos (2\pi/p))\) over \(\mathbf{Q}\)?

\bigskip

\noindent
\textbf{\textbf{Hints}}

Use that \(\mathbf{Q}(\cos(2\pi/p) + i \sin(2\pi/p))\) has degree \((p-1)\) over \(\mathbf{Q}\) and it contains \(\mathbf{Q}(\cos(2\pi/p))\).
\subsection*{Solution}
\label{sec:orge2e9568}
The multiplicative inverse of \(\cos(2\pi/p) + i \sin(2\pi/p)\) is \(\cos(2\pi/p) - i \sin(2\pi/p)\), and the sum of these two numbers is \(2\cos(2\pi/p)\).
So we see that \(\cos(2\pi/p) \in \mathbf{Q}(\cos(2\pi/p) + i \sin(2\pi/p))\).
As a result, we have
\[ \mathbf{Q}(\cos(2\pi/p)) \subset \mathbf{Q}(\cos(2\pi/p) + i \sin(2\pi/p)).\]
The element \(\cos(2\pi/p) + i \sin(2\pi/p)\) satisfies the degree 2 equation
\[ x^2 - 2\cos(2\pi/p)x + 1\]
over the smaller field, so this is at most a quadratic extension.
But it is a non-trivial extension because the bigger field contains non-real numbers.
So this extension has degree 2.
Using that the bigger field has degree \((p-1)\) over \(\mathbf{Q}\), we conclude that the smaller field has degree \((p-1)/2\) over \(\mathbf{Q}\).

\textbf{\textbf{Further question}} (come back to it later)---

What is the degree of \(\mathbf{Q}(\cos (2\pi/p), \sin(2\pi/p))\) over \(\mathbf{Q}\)?
\subsection*{Solution}
\label{sec:orgbd12f26}
It is either the same as the degree of \(\mathbf{Q}(\cos (2\pi/p))\) or twice it, but I am not sure which one.
\section{Most angles cannot be trisected}
\label{sec:orgeb44034}

See if you can prove the following theorem.

\bigskip

\noindent
\textbf{\textbf{Theorem}} ---
Let \(t\) be such that \(\cos t\) is transcendental.
Given \((0,0)\), \((0,1)\), and \((\cos t, \sin t)\), it is impossible to construct \((\cos t/3, \sin t/3)\) using ruler and compass.

\bigskip

\noindent
\textbf{\textbf{Sketch of the proof}}

Follow the same method as in class, keeping track of the field that contains the coordinates of the constructed points.
The starting field will be \(\mathbf{Q}(\cos t, \sin t)\).
The key is to prove that \(\cos (t/3)\) has degree 3 over this field.
It is easier to handle the field \(\mathbf{Q}(\cos t)\), which is isomorphic to \(\mathbf{Q}(x)\), the field of rational functions in a variable \(x\).
Over this field, prove that \(\cos(t/3)\) has degree 3.
To do so, you need to prove that a certain polynomial in \(\mathbf{Q}(x)[y]\) is irreducible.
Using the ideas in class, move through irreducibility in \(\mathbf{Q}(x)[y]\), in \(\mathbf{Q}[x,y]\), and \(\mathbf{Q}(y)[x]\).
Finall conclude that over \(\mathbf{Q}(\cos(t), \sin(t))\) also \(\cos(t/3)\) must have degree 3.
\subsection*{Solution}
\label{sec:org1b78e20}
We follow the same idea as in class.
Recall that the starting point is a field \(F\) that contains the coordinates of our points.
We cannot start with \(F = \mathbf{Q}\), but we take \(F\) to be the smallest subfield of \(\mathbf{C}\) containing \(\cos t\) and \(\sin t\).

How do we describe \(F\)?
It is easier to first look at a smaller field \(G\), whichi s the smallest subfield of \(\mathbf{C}\) containing \(\cos t\).
Convince yourselves that
\[ G = \{p(\cos t)/q(\cos t) \mid p, q \in \mathbf{Q}[x], q \neq 0\}.\]
Furthermore, the map
\[ \mathbf{Q}(x) \to G\]
that sends \(x \mapsto \cos t\) is an isomorphism.

Now \(F = G[\sin t]\) is at most a quadratic extension of \(G\).
The new element \(\sin t\) satisfies the qudratic polynomial
\[ y^2 + \cos^{2} t - 1 =  0\]
(This polynomial is in fact irreducible over \(G\), but we do not need this fact.)

Now, by the triple angle formula, \(\cos(t/3)\) satisfies the equation
\[ 4 y^3 - 3 y - \cos t = 0\]
We claim that this is irreducible over \(F\).
It is easier to see that it is irreducible over \(G\).
Indeed, using the isomorphism above, we can rewrite it as \(4y^3 - 3y + x\).
We now switch to \(\mathbf{Q}[x,y] = \mathbf{Q}[y,x]\), and then to \(\mathbf{Q}(y)[x]\) (why can we do this?)
But in the last ring, it is a linear polynomial and hence irreducible.

We conclude that \(\cos(t/3)\) has degree 3 over \(G\).
Then \(G[\sin t, \cos(t/3)]\) has degree 3 or 6 over \(G\).
In either case, \(\cos(t/3)\) must have degree 3 over \(F = G[\sin t]\).

But then we are done: there is no way to construct \(\cos (t/3)\) starting from \(G\).
\end{document}
