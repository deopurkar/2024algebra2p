% Created 2024-03-05 Tue 20:37
% Intended LaTeX compiler: pdflatex
\documentclass[12pt]{amsart}

\usepackage[utf8]{inputenc}
\usepackage[T1]{fontenc}
\usepackage{graphicx}
\usepackage{longtable}
\usepackage{wrapfig}
\usepackage{rotating}
\usepackage[normalem]{ulem}
\usepackage{amsmath}
\usepackage{amssymb}
\usepackage{capt-of}
\usepackage{hyperref}
\input{workshopformat}
\author{2024 Algebra 2}
\date{}
\title{Workshop 1}
\hypersetup{
 pdfauthor={2024 Algebra 2},
 pdftitle={Workshop 1},
 pdfkeywords={},
 pdfsubject={},
 pdfcreator={Emacs 29.1 (Org mode 9.7-pre)}, 
 pdflang={English}}
\begin{document}

\maketitle
\section{Degree of \(\mathbf{Q}(\cos (2\pi/p))\)?}
\label{sec:org8ed314d}
\noindent
Let \(p\) be a prime number.
What is the degree of \(\mathbf{Q}(\cos (2\pi/p))\) over \(\mathbf{Q}\)?

\bigskip

\noindent
\textbf{\textbf{Hints}}

Use that \(\mathbf{Q}(\cos(2\pi/p) + i \sin(2\pi/p))\) has degree \((p-1)\) over \(\mathbf{Q}\) and it contains \(\mathbf{Q}(\cos(2\pi/p))\).
\section{Most angles cannot be trisected}
\label{sec:orgb33654b}
See if you can prove the following theorem.

\bigskip

\noindent
\textbf{\textbf{Theorem}} ---
Let \(t\) be such that \(\cos t\) is transcendental.
Given \((0,0)\), \((0,1)\), and \((\cos t, \sin t)\), it is impossible to construct \((\cos t/3, \sin t/3)\) using ruler and compass.

\bigskip

\noindent
\textbf{\textbf{Sketch of the proof}}

Follow the same method as in class, keeping track of the field that contains the coordinates of the constructed points.
The starting field will be \(\mathbf{Q}(\cos t, \sin t)\).
The key is to prove that \(\cos (t/3)\) has degree 3 over this field.
It is easier to handle the field \(\mathbf{Q}(\cos t)\), which is isomorphic to \(\mathbf{Q}(x)\), the field of rational functions in a variable \(x\).
Over this field, prove that \(\cos(t/3)\) has degree 3.
To do so, you need to prove that a certain polynomial in \(\mathbf{Q}(x)[y]\) is irreducible.
Using the ideas in class, move through irreducibility in \(\mathbf{Q}(x)[y]\), in \(\mathbf{Q}[x,y]\), and \(\mathbf{Q}(y)[x]\).
Finall conclude that over \(\mathbf{Q}(\cos(t), \sin(t))\) also \(\cos(t/3)\) must have degree 3.
\end{document}
