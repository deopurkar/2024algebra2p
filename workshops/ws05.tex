% Created 2024-05-15 Wed 12:53
% Intended LaTeX compiler: pdflatex
\documentclass[12pt]{amsart}

\usepackage[utf8]{inputenc}
\usepackage[T1]{fontenc}
\usepackage{graphicx}
\usepackage{longtable}
\usepackage{wrapfig}
\usepackage{rotating}
\usepackage[normalem]{ulem}
\usepackage{amsmath}
\usepackage{amssymb}
\usepackage{capt-of}
\usepackage{hyperref}
\usepackage{fullpage}
\author{2024 Algebra 2}
\date{}
\title{Workshop 5}
\hypersetup{
 pdfauthor={2024 Algebra 2},
 pdftitle={Workshop 5},
 pdfkeywords={},
 pdfsubject={},
 pdfcreator={Emacs 29.2 (Org mode 9.7-pre)}, 
 pdflang={English}}
\begin{document}

\maketitle
In this workshop, we will learn how to find Galois groups of irreducible quartic polynomials, up to a small ambiguity.
We fix a base field \(F\) of characteristic 0 and an irreducible \(f(x) \in F[x]\) of degree 4.
Let \(G\) be the Galois group of \(f(x)\).

For your convenience, here is a list of transitive subgroups of \(S_{4}\) with their orders (up to re-numbering).

\begin{center}
\begin{tabular}{lr}
Subgroup & Order\\
\hline
\(S_4\) & 24\\
\(A_4\) & 12\\
\(C_4 = \langle (1234) \rangle\) & 4\\
\(D_4\) & 8\\
\(V = \{e,(12)(34), (14)(23), (13)(24) \}\) & 4\\
\hline
\end{tabular}
\end{center}
\section{Problem 1}
\label{sec:orgc11a661}

Say \(f(x)\) is a quartic with roots \(\alpha_1, \dots, \alpha_4\).
The resolvent cubic \(g(x)\) is the cubic with roots 
\begin{align*}
\beta_1 &= \alpha_1\alpha_2 + \alpha_3\alpha_4\\
\beta_2 &= \alpha_1\alpha_3 + \alpha_2\alpha_4\\
\beta_3 &= \alpha_1\alpha_4 + \alpha_2\alpha_3.
\end{align*}
Check that \(f(x)\) and \(g(x)\) have the same discriminant.
\section{Problem 2}
\label{sec:org7e856fe}

Prove that the discriminant is a square in \(F\) if and only if \(G \subset A_4\).
\section{Problem 3}
\label{sec:org89b3e8f}

Justify the following table (as much as you can) about the Galois group.
Use the following observations.
Let \(F \subset K\) be a splitting field of \(f(x)\).
Let \(L \subset K\) be generated by the 3 roots of the resolvent cubic \(g(x)\).
Then \(F \subset L\) is the splitting field of \(g(x)\).
We have a surjective group homomorphism
\[ \operatorname{Aut}(K/F) \to \operatorname{Aut}(L/F)\]
with kernel \(\operatorname{Aut}(K/L)\).


\begin{center}
\begin{tabular}{lll}
 & Discriminant square & Discriminant non-square\\
\hline
Resolvent irreducible & \(A_4\) & \(S_4\)\\
Resolvent factors as 1+2 & Impossible & \(D_4\) or \(C_4\)\\
Resolvent factors as 1+1+1 & \(V\) & Impossible\\
\hline
\end{tabular}
\end{center}
\end{document}
