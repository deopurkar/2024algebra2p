% Created 2024-05-15 Wed 12:55
% Intended LaTeX compiler: pdflatex
\documentclass[12pt]{amsart}

\usepackage[utf8]{inputenc}
\usepackage[T1]{fontenc}
\usepackage{graphicx}
\usepackage{longtable}
\usepackage{wrapfig}
\usepackage{rotating}
\usepackage[normalem]{ulem}
\usepackage{amsmath}
\usepackage{amssymb}
\usepackage{capt-of}
\usepackage{hyperref}
\usepackage{fullpage}
\author{2024 Algebra 2}
\date{}
\title{Workshop 5}
\hypersetup{
 pdfauthor={2024 Algebra 2},
 pdftitle={Workshop 5},
 pdfkeywords={},
 pdfsubject={},
 pdfcreator={Emacs 29.2 (Org mode 9.7-pre)}, 
 pdflang={English}}
\begin{document}

\maketitle
In this workshop, we will learn how to find Galois groups of irreducible quartic polynomials, up to a small ambiguity.
We fix a base field \(F\) of characteristic 0 and an irreducible \(f(x) \in F[x]\) of degree 4.
Let \(G\) be the Galois group of \(f(x)\).

For your convenience, here is a list of transitive subgroups of \(S_{4}\) with their orders (up to re-numbering).

\begin{center}
\begin{tabular}{lr}
Subgroup & Order\\
\hline
\(S_4\) & 24\\
\(A_4\) & 12\\
\(C_4 = \langle (1234) \rangle\) & 4\\
\(D_4\) & 8\\
\(V = \{e,(12)(34), (14)(23), (13)(24) \}\) & 4\\
\hline
\end{tabular}
\end{center}
\section{Problem 1}
\label{sec:orga0bc632}

Say \(f(x)\) is a quartic with roots \(\alpha_1, \dots, \alpha_4\).
The resolvent cubic \(g(x)\) is the cubic with roots 
\begin{align*}
\beta_1 &= \alpha_1\alpha_2 + \alpha_3\alpha_4\\
\beta_2 &= \alpha_1\alpha_3 + \alpha_2\alpha_4\\
\beta_3 &= \alpha_1\alpha_4 + \alpha_2\alpha_3.
\end{align*}
Check that \(f(x)\) and \(g(x)\) have the same discriminant.
\subsection{Solution sketch}
\label{sec:org9a2846f}
We have \[\beta_2 - \beta_1 = (\alpha_1-\alpha_4)(\alpha_2-\alpha_3)\], and likewise for the other three differences.
So product of the 3 differences of the \(\beta\)'s equals the product of the 6 differences of the \(\alpha\)'s.
\section{Problem 2}
\label{sec:org4308624}

Prove that the discriminant is a square in \(F\) if and only if \(G \subset A_4\).
\subsection{Solution sketch}
\label{sec:org8b3d37c}
We have already seen this in class.
An odd permutation changes the sign of the square root of the discriminant.
So the square root of the discriminant is in \(F\) if and only if there are no odd permutations in \(G\).
\section{Problem 3}
\label{sec:org4fc6139}

Justify the following table (as much as you can) about the Galois group.
Use the following observations.
Let \(F \subset K\) be a splitting field of \(f(x)\).
Let \(L \subset K\) be generated by the 3 roots of the resolvent cubic \(g(x)\).
Then \(F \subset L\) is the splitting field of \(g(x)\).
We have a surjective group homomorphism
\[ \operatorname{Aut}(K/F) \to \operatorname{Aut}(L/F)\]
with kernel \(\operatorname{Aut}(K/L)\).


\begin{center}
\begin{tabular}{lll}
 & Discriminant square & Discriminant non-square\\
\hline
Resolvent irreducible & \(A_4\) & \(S_4\)\\
Resolvent factors as 1+2 & Impossible & \(D_4\) or \(C_4\)\\
Resolvent factors as 1+1+1 & \(V\) & Impossible\\
\hline
\end{tabular}
\end{center}
\subsection{Solution sketch}
\label{sec:orgf6a19b3}

Suppose the resolvent is irreducible and the discriminant is not a square.
Then \(\operatorname{Aut}(L/F) \cong S_3\).
Therefore, we have a surjection \(G \to S_3\).
The only possible \(G\) that could surject onto \(S_3\) are \(S_4\) or \(A_4\) (the order of \(G\) must be divisible by 6).
But since the discriminant is not a square, \(G\) is not \(A_4\), so it must be \(S_4\).

Suppose the resolvent is irreducible and the discriminant is a square.
Then we have a surjection \(G \to A_3\).
So the order of \(G\) is divisible by 3, which leaves the possibilities \(G = S_4\) or \(A_4\).
The discriminant being a square shows that \(G = A_4\).

If the resolvent factors as \(1+2\), then one of the \(\beta\)'s is fixed by \(G\), say \(\beta_1\).
The only permutations that fix \(\beta_1\) are those in \(D_4 \subset S_4\), where we think of \(D_4\) as symmetries of the square with 1, 3 and 2, 4 as opposite vertices.
(Check that the discriminant of a cubic with irreducible factorisation \(1+2\) cannot be a square.)
So \(G \subset D_4\). 
This leaves 3 possibilities \(G = V, C_4, D_4\).
The first one would also fix \(\beta_2\) and \(\beta_3\), which we know is not the case.
So we can strike that off.

Finally, if the resolvent factors as \(1+1+1\), then \(G\) fixes all \(\beta\)'s.
The only such \(G\) is \(G = V\).
\end{document}
