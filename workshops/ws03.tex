% Created 2024-04-16 Tue 20:19
% Intended LaTeX compiler: pdflatex
\documentclass[12pt]{amsart}

\usepackage[utf8]{inputenc}
\usepackage[T1]{fontenc}
\usepackage{graphicx}
\usepackage{longtable}
\usepackage{wrapfig}
\usepackage{rotating}
\usepackage[normalem]{ulem}
\usepackage{amsmath}
\usepackage{amssymb}
\usepackage{capt-of}
\usepackage{hyperref}
\author{2024 Algebra 2}
\date{}
\title{Workshop 3}
\hypersetup{
 pdfauthor={2024 Algebra 2},
 pdftitle={Workshop 3},
 pdfkeywords={},
 pdfsubject={},
 pdfcreator={Emacs 29.2 (Org mode 9.7-pre)}, 
 pdflang={English}}
\begin{document}

\maketitle
Consider the extension \(\mathbf{Q} \subset \mathbf{Q}[e^{2\pi/3}, 2^{1/3}] = F\).
This is a Galois extension, which means that the main theorem of Galois theory applies.

There is an isomorphism
\[ \mathbf{Q}[x,y]/(x^2+x+1, y^3-2) \to \mathbf{Q}[e^{2\pi/3}, 2^{1/3}]\]
that sends \(x\) to \(e^{2\pi i /3/}\) and \(y\) to \(2^{1/3}\).
This is not too hard to prove, but you may proceed without proving it.

\begin{enumerate}
\item Use the presentation above to find all automorphisms of the extension \(F/\mathbf{Q}\).

\item Notice that \(F\) is generated by the three roots of \(x^3-2\).
Prove that any \(\sigma \in \operatorname{Aut}(F/\mathbf{Q})\) must permute the three roots.

\item Label the roots as \(1, 2, 3\).
Then you get a group homomorphism
\[ G  \to S_3.\]
Prove that this is an isomorphism.

\item Using the above, find the subgroup diagram of \(G\).

\item For each subgroup \(H \subset G\), find the fixed field
\[ F^H = \{x \in F \mid \sigma (x) = x \text{ for all } \sigma \in H\}.\]
\end{enumerate}
\end{document}
