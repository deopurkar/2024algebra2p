% Created 2024-03-22 Fri 12:48
% Intended LaTeX compiler: pdflatex
\documentclass[12pt]{amsart}

\usepackage[utf8]{inputenc}
\usepackage[T1]{fontenc}
\usepackage{graphicx}
\usepackage{longtable}
\usepackage{wrapfig}
\usepackage{rotating}
\usepackage[normalem]{ulem}
\usepackage{amsmath}
\usepackage{amssymb}
\usepackage{capt-of}
\usepackage{hyperref}
\author{2024 Algebra 2}
\date{}
\title{Workshop 2}
\hypersetup{
 pdfauthor={2024 Algebra 2},
 pdftitle={Workshop 2},
 pdfkeywords={},
 pdfsubject={},
 pdfcreator={Emacs 29.2 (Org mode 9.7-pre)}, 
 pdflang={English}}
\begin{document}

\maketitle
\section{Factorisation in a finite field}
\label{sec:org15d6025}

The polynomial \(f(x) = x^3+x+1 \in \mathbf{F}_5[x]\) is irreducible.
Let \(K = \mathbf{F}[t]/(f(t))\).
Find the irreducible factorisation of \(f(x)\) in \(K[x]\).
\subsection{Solution sketch}
\label{sec:orgdca28cc}

We know that \(t \in K\) is a root of \(f(x)\).
We can find the other roots by applying the Frobenius, so \(t^3\), and \(t^9\).
So,
\[ f(x) = (x-t)(x-t^3)(x-t^9).\]
Can you bring the roots in the ``reduced form'' (at most quadratics in \(t\))?
\section{Conjugates}
\label{sec:orgc382d12}

Let \(F \subset K\) be a field extension.
We say that \(\alpha, \beta \in K\) are \emph{conjugates} over \(F\) if they have the same minimal polynomial over \(F\).

Let \(K\) be a finite field of characteristic \(p\).
Let \(\phi \colon K \to K\) be the Frobenius map.
\begin{enumerate}
\item Prove that the conjugates of \(a \in K\) are \(a, \phi(a), \phi^2(a), \cdots\).
\item Deduce that the degree of \(a\) over \(\mathbf{F}_p\) is the smallest \(n\) such that \(\phi^n(a) = a\).
\item More generally, let \(K \subset L\) be an extension of finite fields with \(|K| = p^n\).
Prove that the conjugates of \(a \in L\) over \(K\) are \(a, \phi^n(a), \phi^{2n}(a), \dots\).
\item What is the analogue of (2) in this situation?
\end{enumerate}
\subsection{Solution sketch}
\label{sec:orgde88230}

\begin{enumerate}
\item Suppose \(f(x) = \sum a_i x^i\) is the minimal polynomial of \(a\), where \(a_i \in \mathbf{F}_p\).
By applying the Frobenius map, we see that
\[ \phi(\sum a_i a^i) = \sum a_{i} \phi(a)^i = 0,\]
so \(f(\phi(a)) = 0\).
So \(\phi(a)\) also has the same minimal polynomial.

To see that these are \emph{all} the conjugates, let \(n\) be the smallest such that \(\phi^n(a) = a\).
Then \(a, \phi(a), \dots, \phi^{n-1}(a)\) are distinct.
Consider
\[ f(x) = (x-a) \cdots (x-\phi^{n-1}(a)) \in K[x].\]
We see that \(\phi(f) = f\), so \(f \in \mathbf{F}_p[x]\).
In fact, this must be the minimal polynomial of \(a\) (do you see why?).
So the \(\phi^{i}(a)\) are indeed all the conjugates of \(a\).

\item Follows from what we did in (1).

\item This is very similar.  The key idea is that \(a \in L\) lies in \(K\) if and only if \(\phi^n(a) = a\).

\item The degree of \(a \in L\) over \(K\) is the smallest \(m\) such that \(\phi^{nm}(a) = a\).
\end{enumerate}
\section{Factorisation, once again}
\label{sec:org3fbba91}

Let \(f(x) \in \mathbf{F}_p[x]\) be irreducible of degree \(18\).
Let \(\mathbf{F}_p \subset K\) be an extension of degree \(4\).
How does \(f(x)\) factorise in \(K[x]\)?

\noindent
\textbf{Hint}.  Let \(K \subset L\) be an extension of degree 9, so that \(\mathbf{F}_p \subset L\) is of degree 36.  First factorise \(f(x)\) in \(L\) and then ``collect the conjugates'' over \(K\).
\subsection{Solution sketch}
\label{sec:org0327c8b}

Let \(a \in L\) be a root of \(f(x)\).
Then \(18\) is the smallest such that \(\phi^{18}(a) = a\) and the factorisation of \(f(x)\) is
\[ (x-a)(x-\phi(a))\cdots(x-\phi^{17}(a)).\]
The conjugates of \(t \in L\) over \(K\) are \(t, \phi^{4}(t), \phi^{8}(t), \cdots\).
So, the 18 roots \(\phi^i(a)\) split into two sets of conjugates over \(K\), namely \(\phi^i(a)\) for \(i\) even and for \(i\) odd.
This means that \(f(x) \in K[x]\) factorises into a degree 9 irreducible (whose roots are the first set) and another degree 9 irreducible (whose roots are the second set).
\end{document}
