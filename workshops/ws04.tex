% Created 2024-04-30 Tue 15:36
% Intended LaTeX compiler: pdflatex
\documentclass[12pt]{amsart}

\usepackage[utf8]{inputenc}
\usepackage[T1]{fontenc}
\usepackage{graphicx}
\usepackage{longtable}
\usepackage{wrapfig}
\usepackage{rotating}
\usepackage[normalem]{ulem}
\usepackage{amsmath}
\usepackage{amssymb}
\usepackage{capt-of}
\usepackage{hyperref}
\author{2024 Algebra 2}
\date{}
\title{Workshop 4}
\hypersetup{
 pdfauthor={2024 Algebra 2},
 pdftitle={Workshop 4},
 pdfkeywords={},
 pdfsubject={},
 pdfcreator={Emacs 29.2 (Org mode 9.7-pre)}, 
 pdflang={English}}
\begin{document}

\maketitle
In this workshop, we explore the theme of roots, coefficients, and symmetry.

\begin{enumerate}
\item Let
\[ p(x) = x^3 + 2x^2 + 3x + 4,\]
and let \(\alpha, \beta, \gamma \in \mathbf{C}\) be the roots of \(p(x)\).
The expression
\[ \alpha^2 + \beta^2 + \gamma^{2}\]
is symmetric, and hence must be rational.
Find out the exact value.

\item If an expression is not completely symmetric, the more symmetric it is, the ``closer'' it is to the base field.
For example, let \(\alpha, \beta, \gamma\) be the roots (in some big extension) of a cubic \(p(x) \in F[x]\).
Prove that \(\alpha^2 \beta + \beta^2 \gamma + \gamma^2\alpha\) has degree at most 2 over \(F\).\\
\emph{Hint}: Following the idea in the proof of the theorem about splitting fields, try to construct a symmetric polynomial of degree 2 with this as a root.

\item As another application of the principle above, let \(\alpha, \beta, \gamma, \delta\) be the roots (in some extension) of a quartic over \(F\).
Prove that
\[ \alpha\beta + \gamma\delta\]
has degree at most 3 over \(F\).

\item As another application of the principle, let \(\alpha_1, \dots, \alpha_n\) be the roots of \(p(x) \in F[x]\) of degree \(n\).
Consider
\[ d = \prod_{i < j}(\alpha_i - \alpha_j).\]
Prove that \(d\) satisfies a quadratic equation over \(F\).

\item Sometimes, the element is closer to the base-field than we expect from symmetry.
For example, consider the cubic
\[ p(x) = x^3-3x-1.\]
Prove that for this cubic, the element \(d\) above is actually a rational number.
You may find it helpful to consult Wikipedia for the formula for the discriminant of a cubic.
\end{enumerate}
\end{document}
