% Created 2024-04-16 Tue 20:14
% Intended LaTeX compiler: pdflatex
\documentclass[12pt]{amsart}

\usepackage[utf8]{inputenc}
\usepackage[T1]{fontenc}
\usepackage{graphicx}
\usepackage{longtable}
\usepackage{wrapfig}
\usepackage{rotating}
\usepackage[normalem]{ulem}
\usepackage{amsmath}
\usepackage{amssymb}
\usepackage{capt-of}
\usepackage{hyperref}
\author{2024 Algebra 2}
\date{}
\title{Workshop 3}
\hypersetup{
 pdfauthor={2024 Algebra 2},
 pdftitle={Workshop 3},
 pdfkeywords={},
 pdfsubject={},
 pdfcreator={Emacs 29.2 (Org mode 9.7-pre)}, 
 pdflang={English}}
\begin{document}

\maketitle
Consider the extension \(F = \mathbf{Q} \subset \mathbf{Q}[e^{2\pi/3}, 2^{1/3}] = K\).
This is a Galois extension, which means that the main theorem of Galois theory applies.

There is an isomorphism
\[ \mathbf{Q}[x,y]/(x^2+x+1, y^3-2) \to \mathbf{Q}[e^{2\pi/3}, 2^{1/3}]\]
that sends \(x\) to \(e^{2\pi i /3/}\) and \(y\) to \(2^{1/3}\).
This is not too hard to prove, but you may proceed without proving it.

\begin{enumerate}
\item Use the presentation above to find all automorphisms of the extension \(K/F\).

\item Notice that \(K\) is generated by the three roots of \(x^3-2\).
Prove that any \(\sigma \in \operatorname{Aut}(K/F)\) must permute the three roots.

\item Label the roots as \(1, 2, 3\).
Then you get a group homomorphism
\[ G  \to S_3.\]
Prove that this is an isomorphism.

\item Using the above, find the subgroup diagram of \(G\).

\item For each subgroup \(H \subset G\), find the fixed field
\[ K^H = \{x \in K \mid \sigma (x) = x \text{ for all } \sigma \in H\}.\]
\end{enumerate}
\section{Solutions}
\label{sec:org334e57f}

\begin{enumerate}
\item Write \(\zeta_3 = e^{2\pi i /3}\).
 By the presentation, we see that a ring homomorphism \(K \to K\) that fixes \(F\) is specified uniquely by the images \(\alpha, \beta\) of \(x, y\) provided that they satisfy
\[\alpha^2+\alpha+1 = 0 \text{ and } \beta^{3}-2 = 0.\]
The only possibilities are
\begin{align*}
 \alpha &= \zeta_3, \zeta_{3}^{2},\\
 \beta &= 2^{1/3}, \zeta_32^{1/3}, \zeta_3^22^{1/3}.
 \end{align*}
In total, there are 6 possible homomorphism.
All must me automorphisms (why?).

\item If \(\phi \colon K \to K\) is an automorphisms that fixes \(F\) and \(a^3-2 = 0\), then by applying \(\phi\), we see that \(\phi(a)^3-2 = 0\).
So \(\phi\) must send a root to a root.

\item Since the roots generate \(K\), the map must be injective.
Since both sides have the same cardinality, it must be an isomorphism.
It is instructive to take one of the six possibilities and write down the corresponding permutation.
For example, if we label the roots as
\[ \alpha_1 = 2^{1/3},\quad \alpha_2=2^{1/3} \zeta_3,\quad \alpha_3 = 2^{1/3}\zeta_{3}^{2}\]
and take the automorphism given by
\[ 2^{1/3} \mapsto \zeta_32^{1/3} \quad \zeta_3 \mapsto \zeta_3,\]
then the permutation is
\[ \alpha_1 \mapsto \alpha_2 \mapsto \alpha_3 \mapsto \alpha_1.\]

\item The diagram of subgroups of \(S_3\) is
\begin{center}
\includegraphics[width=.6\textwidth]{/home/anand/Documents/teaching/algebra2-2023s1/assets/Course_notes/2023-04-18_13-53-10_screenshot.png}
\end{center}

\item The corresponding diagram of intermediate fields is:

\begin{center}
\includegraphics[width=.6\textwidth]{/home/anand/Documents/teaching/algebra2-2023s1/assets/Course_notes/2023-04-18_13-55-04_screenshot.png}
\end{center}
\end{enumerate}
\end{document}
